\documentclass{article}

% if you need to pass options to natbib, use, e.g.:
%     \PassOptionsToPackage{numbers, compress}{natbib}
% before loading neurips_2020

% ready for submission
% \usepackage{neurips_2020}

% to compile a preprint version, e.g., for submission to arXiv, add add the
% [preprint] option:
\usepackage[preprint]{neurips_2020}

% to compile a camera-ready version, add the [final] option, e.g.:
%     \usepackage[final]{neurips_2020}

% to avoid loading the natbib package, add option nonatbib:
%     \usepackage[nonatbib]{neurips_2020}

\usepackage[utf8]{inputenc} % allow utf-8 input
\usepackage[T1]{fontenc}    % use 8-bit T1 fonts
\usepackage{hyperref}       % hyperlinks
\usepackage{url}            % simple URL typesetting
\usepackage{booktabs}       % professional-quality tables
\usepackage{amsmath}
\usepackage{amssymb}
\usepackage{amsfonts}       % blackboard math symbols
\usepackage{nicefrac}       % compact symbols for 1/2, etc.
\usepackage{microtype}      % microtypography
\usepackage{alltt}
\usepackage{listings}
\usepackage{array}
\usepackage[noline, procnumbered]{algorithm2e}
\usepackage{lmodern}
\usepackage{caption}
\usepackage{subcaption}

\newcommand{\secref}[1]{Section~\ref{#1}}
\newcommand{\tblref}[1]{Table~\ref{#1}}
\newcommand{\figref}[1]{Figure~\ref{#1}}
\newcommand{\thmref}[1]{Theorem~\ref{#1}}
\newcommand{\algref}[1]{Algorithm~\ref{#1}}
\newcommand{\funref}[1]{Function~\ref{#1}}
\newcommand{\listingref}[1]{Listing~\ref{#1}}

\newcommand{\eg}{{\em e.g.}}
\newcommand{\ith}{$i^{th}$}
\newcommand{\cut}[1]{}
\newcommand{\todo}[1]{{\bf\em TODO:} {{\color{red}{#1}}}}

\newcommand{\spd}{\fontfamily{cmr}\textsc{\small StratPD}}
\newcommand{\cspd}{\fontfamily{cmr}\textsc{\small CatStratPD}}
\newcommand{\xnc}{${\bf X}_{\overline{c}}$}
\newcommand{\xnC}{$x_{\overline{C}}$}

%\setlist[enumerate]{itemsep=-1mm}

\title{Technical Report:\\
A Stratification Approach to Partial Dependence for Codependent Variables}

\author{%
  Terence Parr \\
  University of San Francisco\\
  \texttt{parrt@cs.usfca.edu} \\
  \And
  James D. Wilson \\
  University of San Francisco\\
  \texttt{jdwilson4@usfca.edu} \\
}

\begin{document}

\maketitle

\begin{abstract}
\end{abstract}

\section{Introduction}

partial dependence is important because...

Existing techniques, such as FPD, ICE, ALE, SHAP peer through the lens of a model's predictions. For the same data applying the same technique but using different models, we get different answers, which calls into question the validity of the curves.

key is "all else being equal", which implies you don't want curves affected by other variables. Interaction plots are also very useful, such as ICE, but here our goal is the pure partial dependence curve. In the future, we hope to consider extracting interaction between variables like SHAP.

Many analysts do not need a predictive model nor would they know how to choose, tune, and assess a model. Could also be the case that a technique is not available in the desired deployment environment.  The techniques differ in algorithm simplicity, performance, and ability to isolate codependent variables. a nonparametric technique could also inform which machine learning model to use if a model is desired.

we introduce an ideal definition of partial dependence that does not rely on predictions from a fitted model based upon partial derivatives and then estimate partial derivatives nonparametrically to get partial dependence. The technique seems to isolate variables well and has linear behavior for numeric variables and mildly quadratic behavior for categorical variables in practice. The theoretical complexity is $O(n^2)$ like FPD.

SHAP is mean centered FPD for independent variables, proof in supplemental material.

state up front it only gets pure partial dependence, no interaction and has quadratic theoretical complexity, but it has the advantage that it doesn't require a fitted model.  Sometimes there is an advantage to a model, smoothing etc. But, in many cases lack of model increases the accessibility of the tool to analysts and could prevent nonexpert machine learning practitioners from interpretation errors from poorly fit or tuned models.

\section{Partial dependence without model predictions}

\noindent {\bf Definition 1} The {\em ideal partial dependence} of $y$ on feature $x_j$ for smooth generator function $f:\mathbb{R}^{p} \rightarrow \mathbb{R}$ evaluated at $x_j = z$ is the cumulative sum up to $z$:

\begin{equation}\label{eq:pd}
\text{\it PD}_j(z) = \int_{min(x_j)}^z \frac{\partial y}{\partial x_j} dx_j
\end{equation}

$\text{\it PD}_j(z)$ is the value contributed to $y$ by $x_j$ at $x_j = z$ and $\text{\it PD}_j(min(x_j))=0$. The advantages of this partial dependence definition are that it does not depend on predictions from a fitted model and is insensitive to collinear or otherwise codependent features, unlike the Friedman's original definition that he points out is less accurate for codependent data sets. We will denote Friedman's as $\text{\it FPD}_j$ to distinguish it from this ideal, $\text{\it PD}_j$.

For example, consider quadratic equation $y = x_1^2 + x_2 + 100$ as a generator of data in $[0,3]$. The partial derivatives are $\frac{\partial y}{\partial x_1} = 2 x_1$ and $\frac{\partial y}{\partial x_2} = 1$, giving $\text{\it PD}_1 = x_1^2$ and $\text{\it PD}_2 = x_2$. 

The obvious disadvantage of this feature impact definition is that function $f$, from which $\text{\it PD}_j$ is derived, is unknown in practice, so symbolically computing the partial derivatives is not possible. But, if we could compute accurate partial dependence curves by some other method, then this definition would still represent a viable means to obtain feature impacts. 

\spd{} stratifies a data set into groups of observations that are similar, except in the variable of interest, $x_j$, through the use of a single decision tree. Any fluctuation of the response variable within a group (decision tree leaf) is likely due to $x_j$.  The $\beta_1$ coefficient of a simple local linear regression fit to the $(x_j, y)$ values within a group provides an estimate of $\frac{\partial y}{\partial x_j}$ in that group's $x_j$ range. Averaging the partial derivative estimates across all such groups yields the overall $\frac{\partial y}{\partial x_j}$ partial derivative approximation. The cumulative sum of the estimated partial derivative yields the partial dependence curve. 

\section{Existing work}

FPD

ICE

ALE

SHAP
\pagebreak
\section{Algorithms}


\setlength{\algomargin}{5pt}
\begin{algorithm}[]
\DontPrintSemicolon
\SetAlgorithmName{Algorithm}{List of Algorithms}
\SetAlgoSkip{}
\TitleOfAlgo{{\em StratPD}}
\KwIn{{\bf X}, {\bf y}, c, {\it min\_samples\_leaf}, {\it min\_slopes\_per\_x}}
\KwOut{$\bf pdx, pdy$: Unique $x_c$, partial dependence values across $x_c$}
Train decision tree regressor $T$ on (\xnc{}, $\bf y$) with hyper-parameter: ${\it min\_samples\_leaf}$\;
\For{each leaf $l \in T$}{
        $({\bf x}_l, {\bf y}_l)$ = $\{(x_c^{(i)},  y^{(i)})\}_{i \in l}$\\
        {\bf ux} := $unique({\bf x}_l)$\\
	Group leaf records $({\bf x}_l, {\bf y}_l)$ by $x_c$, computing $\bar{y}$ per unique $x_c$\\
	${\bf dx}$ := ${\bf ux}^{(i+1)} - {\bf ux}^{(i)}_{i=1..|{\bf ux}|-1}$\tcp*{\it Discrete difference}
	${\bf dy}$ := $\bar{\bf y}^{(i+1)} - \bar{\bf y}^{(i)}_{i=1..|{\bf ux}|-1}$\\
	Add tuples ${({\bf ux}^{(i)}, {\bf ux}^{(i+1)},~ {\bf dy}^{(i)}/{\bf dx}^{(i)})}_{i=1..|{\bf ux}|-1}$ to list ${\bf d}$\\
}
$\bf ux$ := $unique(\{x_c^{(i)}\}_{i=1..n})$\tcp*{\it Compute average slope per unique $x_c$ value}
\For{each $x \in {\bf ux}$}{
	$slopes$ := [$slope$ for $(a,b,slope) \in \bf d$ if $x \ge a$ and $x <b$]\\
	${\bf c}_x$ := $|slopes|$, ${\bf dydx}_x$ := $\overline{slopes}$\\
}
${\bf dy}$ := ${\bf dy}[{\bf c} \ge min\_slopes\_per\_x]$\tcp*{\it Drop slope estimates computed from too few}
${\bf ux}$ := ${\bf ux}[{\bf c} \ge min\_slopes\_per\_x]$\\
$\bf pdx$ := ${\bf ux}^{(i+1)} - {\bf ux}^{(i)}_{i=1..|{\bf ux}|-1}$\\
$\bf pdy$ := [0] + cumulative\_sum(${\bf dydx} * \bf pdx$)~~~\tcp*{\it integrate, inserting 0 for leftmost $x_c$}
\Return{$\bf pdx, pdy$}
\label{alg:CatStratPD}
\end{algorithm}


\setlength{\algomargin}{5pt}
\begin{algorithm}[]
\DontPrintSemicolon
\SetAlgorithmName{Algorithm}{List of Algorithms}
\SetAlgoSkip{}
\TitleOfAlgo{{\em CatStratPD}}
\KwIn{$\begin{array}[t]{l}
{\bf X}, {\bf y}, c,\\
{\it n_trials}, {\it min\_samples\_leaf}\\
\end{array}$
}
\KwOut{$\begin{array}[t]{l}
\Delta^{(k)} = \text{category } k \text{'s effect on } y \text{ where } mean(\Delta^{(k)})=0\\
n^{(k)} = \text{number of supported observations per category $k$}\\
\end{array}$
}
\label{alg:CatStratPD}
\end{algorithm}

\section*{References}

\bibliography{stratpd}
\end{document}